%% CODE HIGHLIGHTING LaTeX file, DO NOT TOUCH
%% This file should only be edited by someone experienced with LaTeX, 
%% if you don't know what you're doing then I recommend you stay your hand.
%% All content is placed in the ``sections'' folder, this is probably where 
%% you want to go.

%% Primitive syntax rendering
\usepackage{fancyvrb}

%% More fancy syntax rendering
\usepackage{listings}
%% lstautogobble is a custom package, that removes leading spaces in 
%% lstlistings environments.
\usepackage{lstautogobble}

\usepackage{xcolor}

\renewcommand{\lstlistingname}{Code snippet}

\newif\ifsyntaxcolor
\syntaxcolortrue
\ifsyntaxcolor
\definecolor{code-back}{RGB}{249,245,249}
\definecolor{code-comment}{RGB}{57,93,161}
\definecolor{code-keyword}{RGB}{127,0,85}
\definecolor{code-number}{RGB}{51,51,51}
\definecolor{code-string}{RGB}{51,151,51}
\else
\definecolor{code-back}{rgb}{1,1,1}
\definecolor{code-comment}{rgb}{0.5,0.5,0.5}
\definecolor{code-keyword}{rgb}{0.5,0.5,0.5}
\definecolor{code-number}{rgb}{0.5,0.5,0.5}
\definecolor{code-string}{rgb}{0.5,0.5,0.5}
\fi

% macro to select a scaled-down version of Bera Mono (for instance)
%\makeatletter
%\newcommand\BeraMonottfamily{%
%	\def\fvm@Scale{0.75}% scales the font down
%	\fontfamily{fvm}\selectfont% selects the Bera Mono font
%}
%\makeatother

\usepackage[scaled=0.75]{beramono}

\lstdefinestyle{basic}{
	%% Syntax color
	backgroundcolor=\color{code-back},
	commentstyle=\color{code-comment},
	keywordstyle=\color{code-keyword},
	numberstyle=\color{code-number},
	stringstyle=\color{code-string},
	basicstyle=\ttfamily,
	%% Line numbers
	numbers=left,
	numbersep=5pt,
	frame=leftline,
	%% Place caption
	captionpos=b,
	%% Hide spaces in strings
	showstringspaces=false,
	%% Tabsize
	tabsize=3,
	%% Allow æøå
	extendedchars=true,
	literate=%
		{æ}{{\ae}}1
		{å}{{\aa}}1
		{ø}{{\o}}1
		{Æ}{{\AE}}1
		{Å}{{\AA}}1
		{Ø}{{\O}}1,
	autogobble
}

\newcommand{\TS}{\lstset{ style=TS }}
\newcommand{\TSS}{\lstset{ style=TSS }}

\newcommand{\TSVariables}{\%obj1,\%obj2}

\lstdefinelanguage{TorqueScript}{
	showstringspaces=false
	sensitive=false,
	otherkeywords = {\%,\%this},
	keywords=[0]{function, datablock, delete, messageboxok, exec, echo, getcount, commandtoclient, commandtoserver, schedule, if, else, new, getObject, addObject,
		singleton, bind, getRowNumById, addRow, sortNumerical, clearSelection,setRowById,removeRowById,getWord,setWord,getWordCount,getField,setField,
		StripMLControlChars},
	keywords=[1]{StaticShapeData, ParticleEmitterData, ParticleEmitterNodeData, ParticleData, ParticleEmitterNode,ClientGroup, SimGroup, SimObject,
		GuiControl, GuiBitmapBorderControl, GuiBitmapControl, GuiControlProfile, GuiTextListCtrl, GuiScrollCtrl, GuiTextCtrl, GuiPanel},
	keywords=[2]{SPC,@,TAB},
	keywords=[3]{\$CoinsFound},
	keywords=[4]{\%,obj,obj1,\%obj2,col,vec,len,score,emitterNode,idx,idxClient,winnerClient,guiContent,val,score,kills,text,name,deaths,clientId,clientName,msgType,msgString},
	morekeywords=[5]{\%this},
	keywords=[5]{\%this},
	keywordstyle=[0]{\color{Blue}},
	keywordstyle=[1]{\color{DarkBlue}},
	keywordstyle=[2]{\color{LightGreen}},
	keywordstyle=[3]{\color{Orange!100}},
	keywordstyle=[4]{\color{CornflowerBlue!80}},
	keywordstyle=[5]{\color{blue}\%},
	morestring=[s][\color{code-string}]{"}{"},
	morestring=[s][\color{code-string}]{'}{'},
	morecomment=[l][\color{code-comment}]{//},
	morecomment=[s][\color{code-comment}]{/*}{*/},
	escapeinside={-*}{*)},
}

%% Non-floating JavaCode
\lstnewenvironment{JavaCodeH}[2]
	{\lstset{language=Java, style=basic, caption={#1}, label=#2}\noindent}
	{}

\lstnewenvironment{JavaCode}[2]
	{\lstset{language=Java, style=basic, caption={#1}, label=#2, float, floatplacement=H}\noindent}
	{}
	
\newcommand{\JavaInline}{\lstset{language=Java, style=basic}\lstinline}
	
\lstnewenvironment{XmlCode}[2]
	{\lstset{language=XML, style=basic, caption={#1}, label=#2, float, floatplacement=H}\noindent}
	{}
	
\newcommand{\XmlInline}{\lstset{language=Xml, style=basic}\lstinline}
	
\lstnewenvironment{LaTeXCode}[2]
	{\lstset{language=[LaTeX]TeX, style=basic, caption={#1}, label=#2, float, floatplacement=H}\noindent}
	{}

\newcommand{\LaTeXInline}{\lstset{language=[LaTeX]TeX, style=basic}\lstinline}

\lstnewenvironment{TSCode}[2]
{\lstset{language=TorqueScript, style=basic, caption={#1}, label=#2, float}\noindent}
{}

\lstnewenvironment{TSCodeH}[2]
{\lstset{language=TorqueScript, style=basic, caption={#1}, label=#2, float, floatplacement=H}\noindent}
{}

\newcommand{\TSInline}{\lstset{language=TorqueScript, style=basic}\lstinline}

	