% !TeX spellcheck = en_GB


\chapter{Setting Up Torque3D}
\label{cha:t3d-setup}

In Torque3D, you generally refer to \textit{scripting} when you write the game-logic using TorqueScript or \texttt{C\#}. On the other hand, when you write \texttt{C++} in the engine itself, it is generally referred to as \textit{coding}.

It might seem strange at first, as you might say that you are coding in \texttt{C\#}, but the two different terms makes it easier to converse about Torque3D in day-to-day language.

If you are only looking to be a scripter, you can find pre-compiled binaries for Torque3D. However, while scripting is great to get started with the engine, you will probably quickly want to learn how to compile the engine as well.

In this section, I will first cover how to compile the engine and then how to get started scripting. 

\begin{remark}
	This chapter assumes you are using Windows, if you need to set Torque3D up for Linux I recommend you seek help from the community.
\end{remark}

\section{Compiling the engine}
\subsection{Prerequisites}

In order to compile the engine, you will need the following tools installed on your computer:
\begin{description}
	\item[CMake] is used to generate the project files for Visual Studio. You can download it at \url{https://cmake.org/download/}
	\item[Visual Studio] is necessary in order to compile the engine.
	Visit \url{https://visualstudio.microsoft.com/downloads/} to download the latest version.
	\item[Windows SDK] is a library that contains DirectX and other necessary libraries for compiling the engine. You can download it here: \url{https://developer.microsoft.com/en-US/windows/downloads/windows-10-sdk/}
\end{description}

\subsection{Downloading the source}
\blindtext

\subsection{Configuring CMake}
\blindtext

\subsection{Compiling in Visual Studio}
\blindtext

\section{Adding C\#}
C\# is not native to the engine, it is an optional add-on. In order for it to work, a lot of code-generation is necessary.

\blindtext

\subsection{Downloading the framework}
\blindtext

\subsection{Running the code-generator}
\blindtext


\section{Creating the CoinCollection module}

\subsection{Directory layout}
\blindtext

\subsection{The C\# Project}
\blindtext

